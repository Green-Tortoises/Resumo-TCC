\section{Materials} \label{Materials}

This work is being developed in the C++ language, and specific libraries have been chosen for the development of this application. The ROCm library\cite{rocm} and rocRAND\cite{rocrand} are the most suitable tools for this situation.
The first is a C++ library designed to program applications for AMD and Nvidia GPUs\footnote{Check support for specifics graphics card and CPUs at \href{https://rocm.docs.amd.com/en/latest/release/gpu_os_support.html}{ROCm - GPU and OS Support}} for data processing, very suitable for large matrix math.
The second is a library that supports pseudo random numbers generators inside kernel functions. By using both together, it is possible to develop a performant matrices application on GPUs.
Additionally, four public databases have been chosen. All databases have only numeric attributes since ACO works only with numerical values.

\begin{figure}
    \centering
    \begin{tabular}{|c|c|c|}
        \hline
        Database & Number of Instances & Link to Database \\
        \hline
        Haberman & 230 & \href{https://raw.githubusercontent.com/LucasSnatiago/Ant-Colony-ROCm/main/database/haberman_treino_normal.csv}{Ant-Colony-ROCm: Haberman} \\
        Optdigits & 4215 & \href{https://raw.githubusercontent.com/LucasSnatiago/Ant-Colony-ROCm/main/database/optdigits_treino_normal.csv}{Ant-Colony-ROCm: Optdigits} \\
        Mushroom & 6093 & \href{https://raw.githubusercontent.com/LucasSnatiago/Ant-Colony-ROCm/main/database/mushrooms_treino_normal.csv}{Ant-Colony-ROCm: Mushroom} \\
        Body Performance & 10044 & \href{https://raw.githubusercontent.com/LucasSnatiago/Ant-Colony-ROCm/main/database/body_performance_treino_normal.csv}{Ant-Colony-ROCm: Body Perfomance} \\
        \hline
    \end{tabular}
    \caption{Databases used for measuring performance.}
    \label{fig:datasets}
\end{figure}

These databases were deliberately chosen with significantly different numbers of instances since this algorithm has exponential growth in data processing cost. Starting with small databases for initial testing implies considerably shorter processing times. For each new test, increasing the number of instances significantly raises the processing cost and execution time, which can be used to test the efficiency of the parallel version of the algorithm.

All these tests are being conducted on a computer running Arch Linux, Kernel 6.0.10-zen2-1-zen. The machine has a Ryzen 5 3600xt CPU and an AMD Radeon 7600 GPU with 8GB of VRAM, along with 16GB of RAM.

These tools were chosen for their simplicity, considering that C++ is a straightforward programming language. The databases were selected due to their varying numbers of instances, as mentioned earlier. Regarding hardware, it is necessary to have any GPU that supports ROCm\footnote[2]{Check support at \href{https://rocm.docs.amd.com/en/latest/release/gpu_os_support.html}{GPU Support and OS Compatibility}.}.
