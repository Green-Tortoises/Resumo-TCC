\section{Materials} \label{Materials}

This work is being developed in the C++ language, and specific libraries have been chosen for the development of this application. The ROCm library\cite{cupyLib} and Numpy\cite{numpyLib} are the most suitable tools for this situation. The first is a Python library designed to support Nvidia GPUs\footnote[1]{There is a version of this library for ROCm to support both Nvidia and AMD GPUs, which will be used in this work.} for large matrix calculations. The second is a library that supports multidimensional arrays (but only on the CPU). By using both together, it is possible to develop a performant application using multidimensional arrays on GPUs.

Additionally, three public databases have been chosen. All databases that originally contained categorical attributes had their attributes converted into numerical attributes since ACO works only with numerical attributes.

\begin{figure}
    \begin{center}
        \begin{tabular}{|c|c|c|}
            \hline
            Database & Number of Instances & Link to Database \\
            \hline
            Haberman & Number of Instances & Link to Database \\
            \hline
            Optdigits & Number of Instances & Link to Database \\
            \hline
            Mushroom & Number of Instances & Link to Database \\
            \hline
            Body Performance & Number of Instances & Link to Database \\
            \hline
        \end{tabular}
        \caption{Databases used for measuring performance.}
    \end{center}
    \label{fig:datasets}
\end{figure}

These databases were deliberately chosen with significantly different numbers of instances since this algorithm has exponential growth in data processing cost. Starting with small databases for initial testing implies considerably shorter processing times. For each new test, increasing the number of instances significantly raises the processing cost and execution time, which can be used to test the efficiency of the parallel version of the algorithm.

All these tests are being conducted on a computer running Arch Linux, Kernel 6.0.10-zen2-1-zen. The machine has a Ryzen 5 3600xt CPU and an AMD Radeon 7600 GPU with 8GB of VRAM, along with 16GB of RAM.

These tools were chosen for their simplicity, considering that Python is a straightforward programming language. The databases were selected due to their varying numbers of instances, as mentioned earlier. Regarding hardware, it is necessary to have any GPU that supports ROCm\footnote[2]{Check the support at \href{https://rocm.docs.amd.com/en/latest/release/gpu_os_support.html}{GPU Support and OS Compatibility}.}.
