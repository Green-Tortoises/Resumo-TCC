\section{Final Thougths}

The most appropriate parallelization technique is ultimately dependent upon the
nature of the problem being solved. The approach adressed by this paper
is extremely scalable being able to grow as much as the number of threads available on the GPU.
However this is not a fully implementation of the Ant Colony Optimization algorithm,
removing the pheromones it is just a "muted" ant. There is no communication between the ants.
Considering this fact, this algorithm can be considered an optimized brute-force.

Removing the pheromones, this algorithm removes the data dependency between the ants.
This reduce the huge overhead added by waiting for the ants to communicate then each GPU
thread can calculate the best path using randomness independently. This approach made this algorithm
very scalable, being limited only by the memory available. In the end, after all ants
being processed, it's just a matter of choosing the best one between all ants in the colony.

The memory management is the biggest scalability issue to be addressed in a future paper. This algorithm
takes a lot of the device memory, being at least $\Omega({n^{2})}$ in the best scenario.
This memory use was designed to be that way for better perfomance, but it makes impossible to run big datasets, running out
of memory really fast. A better proposal is to use less memory even though it can take a little bit of a perfomance hit, but
letting it doable to run bigger datasets.
