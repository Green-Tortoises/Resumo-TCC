\subsection{Related Work}

% I'm hardcoding the citations because SBC Template is broken

As presented by \cite{ACONewAlgorithm} in ``Ant colony optimization: a new meta-heuristic'', several
new algorithms inspired by the behavior of ant colonies are being used
to solve discrete optimization problems. Several
examples of applications are presented in that article, one example is
the travelling salesman problem.

Years later, some of the authors mentioned earlier came together again and
wrote a new article called ``Ant colony optimization: Introduction and recent trends'', summarizing all the updates
and modifications that occurred during this period in the algorithm. \cite{UpdatesInACO} introduces
some new problems that have been incorporated into the ACO, such as the clique problem and problems
related to protein binding that could be solved with a slight variation of this algorithm.

The authors, as detailed in \cite{ACONewAlgorithm20anos},
created a platform aimed at educating new researchers in the ant colony optimization algorithm.
This platform features the most recent iteration of the algorithm
\footnote{The platform is available through this link: \url{http://www.scholarpedia.org/article/Ant_colony_optimization}}.
Upon accessing it, you will find the algorithm's definition, its formal definition, a metaheuristic
of the algorithm, and, finally, the applications that are using this method the most and the new trends
that this algorithm is following.

\cite{paralellAco} showed a different approach to achieve parallelism in the ACO algorithm.
In their paper, he created an algorithm that place all the ants in different
cities getting up to three times speed up over the serial version of the code. This version only works
using CPU threads and it doesn't scale up for dozens of threads.

The last important paper to be presented is \cite{openmpAco}. This versions uses OpenMP as the library
to run the ACO in parallel using a Silicon Graphics Origin 2000 machine.
It uses a quadtree to store the values of pheromones after each execution.
The major challenge of this approach lies in the requirement to synchronize
pheromones among all ants after each batch completes its loop. This necessitates pausing the
execution of all threads during the data synchronization phase for each batch of ants.
