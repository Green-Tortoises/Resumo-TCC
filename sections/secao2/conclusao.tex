Neste artigo será usado \emph{CUDA} para otimizar o algoritmo \emph{Ant Colony Optimization} usando a biblioteca
\emph{CuPy} \cite{Cupy} que permite o uso de cálculos de matrizes em GPUs Nvidia. Essa biblioteca
tem como objetivo acelerar o processamento de arranjos multidimensionais, com isso torna-se perfeita
para todos os algoritmos já citados neste artigo.

Será usado o \emph{K-Means} como unidade comparativa deste trabalho para mensurar 
o ganho que a \emph{Ant Colony Optimization} é capaz de obter, tanto quanto sua acertividade
em encontrar o melhor caminho. Aproveitando de todas as otimizações que já foram citadas previamente
e todos os trabalhos que foram feitos para tornar o Python a ferramenta mais procurada
para pesquisadores de inteligências artificiais. Com isso, esse trabalho tem como objetivo
avançar as pesquisas previamente feitas para testar a eficácia de uma versão otimizada em 
GPU do algoritmo ACO.