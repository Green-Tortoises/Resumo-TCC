\subsection{Conclusão}

Todos os trabalhos anteriomente citados estão presos dentro do contexto de usar seleção 
de instâncias com \emph{Ant Colony Optimizations} de forma sequencial tornando-o o mais eficiente 
possível para ser executado apenas por uma \emph{thread}. Neste trabalho será modificado alguns dos 
parâmetros da aplicação para permití-la executar em várias \emph{threads} da GPU. Isso deve tornar 
o algoritmo mais eficiente para todos os usos já citados pelos outros autores.

Será usado o \emph{K-Means} como unidade comparativa deste trabalho para mensurar 
o ganho que a \emph{Ant Colony Optimization} é capaz de obter, tanto quanto sua acertividade
em encontrar o melhor caminho. Aproveitando de todas as otimizações que já foram citadas previamente
e todos os trabalhos que foram feitos para tornar o Python a ferramenta mais procurada
para pesquisadores de inteligências artificiais. Com isso, esse trabalho tem como objetivo
avançar as pesquisas previamente feitas para testar a eficácia de uma versão otimizada em 
GPU desse algoritmo ACO.
