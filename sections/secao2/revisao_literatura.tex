\section{Revisão da Literatura}

O livro \cite{BookIA} apresenta todos os conceitos necessários para se iniciar na área de inteligência artificial (IA).
Modelos de IA são pensados desde os primórdios da computação. Sempre foi almejado uma forma de se construir um algoritmo
que se auto-modela dependendo do problema e que consegue se auto-treinar e auto-melhorar precisando apenas de uma entrada
maior.

Como citado por \citeauthor{AntColonyOptimization} \cite{AntColonyOptimization}, algoritmos baseados em colônias de formigas
e outros insetos, que possuem comportamentos parecidos, começaram a ser pensados ainda nos anos noventa. 
Percebendo detalhes na forma que formigas se organizam ao procurar melhores
caminhos para alcançar seus objetivos, inspirou vários pesquisadores a projetarem algoritmos que o imitem como forma
de otimização para a área de inteligência artificial.

Programar qualquer inteligência artificial possui o problema de sempre se preocupar com o tamanho da entrada para treinamento.
Uma IA que sempre precisa de grandes entradas para se ter bons resultados precisa de ter boa performance por elemento presente nessa entrada. 
Por conta disso modelar esse programa usando de
computação paralela é fundamental para um bom uso dos recursos da máquina. A diferença de desempenho entre a versão sequêncial
para a versão paralela é tão significativo que projetar os algoritmos apenas de forma paralela pode fazer considerável sentido
\cite{SequentialVSParallel}.