\subsection{Resultados Esperados} \label{resultadosEsperados}

É esperado que ao fim desse trabalho seja possível identitificar uma melhora de desempenho
no algoritmo ACO. Considerando que placas de vídeo são otimizadas para equações matriciais e
é possível trabalhar essas equações de forma eficiente dentro de GPUs. Melhor desempenho,
permitiria, então, processar maiores bases de dados.

Ao fim do trabalho é esperado que a última base de dados possa ser executada com um tempo
consideravelmente menor do que sua versão sequencial. A menor base de dados não é esperada
para ter grandes ganhos de desempenho, uma vez que já é uma base bem reduzida e sua execução
acontece em poucos segundos mesmo que de forma sequencial.

O crescimento do algoritmo ACO que está sendo trabalhado nessa monografia é exponencial.
Ao torná-lo paralelo a equação que define seu crescimento terá uma curva de crescimento mais lento,
crescendo de forma exponencial, mas sendo capaz de executar uma base de dados superior no mesmo tempo
que uma base menor executando de forma sequencial. Isso poderá ser medido por gráficos contendo
o número de instâncias sendo executadas pelo tempo.
