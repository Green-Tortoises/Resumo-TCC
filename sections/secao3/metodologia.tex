\section{Metodologia}

Para se alcançar os resultados almejados foram selecionados as ferramentas CuPy e Numpy que serão citadas
na próxima sessão. 

\subsection{Materiais}

Este trabalho está sendo desenvolvido na linguagem Python e foram escolhidos bibliotecas 
específicas para o desenvolvimento dessa aplicação.
A biblioteca CuPy\cite{cupyLib} e Numpy\cite{numpyLib} são as ferramentas mais adequadas
para essa situação. A primeira é uma biblioteca escrita em Python para suportar GPUs Nvidia
para grandes cálculos matriciais. A segunda é uma biblioteca que suporta matrizes multidimensionais
(possuí suporte apenas para a CPU). Ao usar as duas em conjunto é possível desenvolver uma aplicação
performática usando matrizes multidimensionais em placas de vídeo.

\subsection{Cronograma}

Esses objetivos serão desenvolvidos ao longo do fim de 2022 e início de 2023, conforme o 
cronograma a seguir:

\begin{center}
    \begin{tabular}{|c|c|}
        \hline
        Mês & O que será produzido \\
        \hline
        Novembro & Finalizar a primeira versão deste documento \\
        \hline
        Dezembro & Fazer alguns testes de como ficaria o algoritmo em \emph{batches} \\
        \hline
        Janeiro & Não será produzido nada \\
        \hline 
        Fevereiro & Escolher alguma abordagem \\
        \hline
        Março & Programar o modelo \\
        \hline
        Abril & Contabilizar as métricas \\
        \hline
        Maio & Escrever a metodologia deste documento \\
        \hline
        Junho & Preparar apresentação e revisar documento \\
        \hline
    \end{tabular}
\end{center}



\subsection{Resultados Esperados}

É esperado que ao fim desse trabalho seja possível identitificar uma melhora de desempenho
no algoritmo ACO. Considerando que placas de vídeo são otimizadas para equações matriciais e
é possível trabalhar essas equações de forma eficiente dentro de GPUs, pode-se ter ganhos consideráveis
de desempenho. 