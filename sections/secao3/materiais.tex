\subsection{Materiais} \label{Materiais}

Este trabalho está sendo desenvolvido na linguagem Python e foram escolhidos bibliotecas 
específicas para o desenvolvimento dessa aplicação.
A biblioteca CuPy\cite{cupyLib} e Numpy\cite{numpyLib} são as ferramentas mais adequadas
para essa situação. A primeira é uma biblioteca escrita em Python para suportar GPUs Nvidia
para grandes cálculos matriciais. A segunda é uma biblioteca que suporta matrizes multidimensionais
(possuí suporte apenas para a CPU). Ao usar as duas em conjunto é possível desenvolver uma aplicação
performática usando matrizes multidimensionais em placas de vídeo.

Foram escolhidas, também, três bases de dados públicas.
Todas as bases de dados que, originalmente continham atributos categóricos, tiveram os
seus atributos convertidos em atributos numéricos, uma vez que o ACO trabalha
somente com atributos numéricos.

\begin{center}
    \begin{tabular}{|c|c|c|}   
        \hline
        Base de Dados & Número de Instâncias & Link Para a Base de Dados \\
        \hline
        Banana & 5300 & \href{https://www.kaggle.com/datasets/saranchandar/standard-classification-banana-dataset}{Link para o Banana Dataset} \\
        \hline
        \emph{Cyber Security} & 1247 & \href{https://www.kaggle.com/datasets/deepcontractor/cyber-security-salaries}{Link para o \emph{Cyber Security}} \\
        \hline
        \emph{Iris Species} & 150 & \href{https://www.kaggle.com/datasets/uciml/iris}{Link para o \emph{Iris Species}} \\
        \hline 
    \end{tabular}
\end{center}

Essas bases de dados foram escolhidas com quantidades de instâncias bem diferentes propositalmente, uma vez
que esse algoritmo possuí um crescimento exponencial no custo de processamento dos dados. Para um teste inicial
começar com bases pequenas implica em tempo consideravelmente pequeno de processamento. Para cada novo teste
aumentar o número de instâncias aumenta consideravelmente o custo de processamento e o tempo de execução que 
podem ser usados para testar a eficiencia da versão paralela do algoritmo.

Todos esses testes estão sendo feitos em um computador com Arch Linux, Kernel \emph{6.0.10-zen2-1-zen}.
A máquina possuí um \emph{Ryzen 5 3600xt} e uma GPU \emph{GeForce GTX 1060 3GB}, com 16GB de memória RAM.

Foi escolhido essas ferramentas pensando na simplicidade, uma vez que Python é uma língua de programação
bem simples. As bases de dados foram selecionadas por conterem um número bem diferente de instâncias, assim
como já havia sido citado anteriormente. Para o \emph{hardware} é necessário ter qualquer placa de vídeo da 
Nvidia, pois CUDA funciona apenas nesses modelos.
