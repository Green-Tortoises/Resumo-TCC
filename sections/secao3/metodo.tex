\subsection{Método} \label{Metodo}

Usando das ferramentas citadas, será necessário construir um algoritmo modificado
do ACO para ter-se a capacidade de executá-lo de forma paralela em GPUs.
O primeiro modelo de algoritmo pensado foi calcular o movimento das formigas em \emph{batches}.
Com isso, pode-se tratar um conjunto grande de formigas simultaneamente de uma só vez. 
Quando essas formigas terminarem de procurar o melhor caminho, elas voltam, processa-se todos os 
dados e envia mais formigas para calcular o novo melhor caminho e assim por diante, até que toda 
a base de dados seja processada.

Será necessário contabilizar o tempo que cada base de dados está tomando para processar todos os 
caminhos. Com isso, será produzido vários gráficos contendo o número de instâncias que estão sendo 
processadas pelo tempo de execução. Vendo esses gráficos será possível descobrir onde deve-se otimizar
as funções criadas para produzir versões paralelas melhores.

Por fim, terá as métricas mais otimizadas que foram possíveis de conseguir. Os ganhos serão todos 
medidos e apresentados nesse documento. O algoritmo final obtido, também será apresentado para que 
possa ser replicado e testado.