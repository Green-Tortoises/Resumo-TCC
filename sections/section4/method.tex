\subsection{Method} \label{Method}

Using the mentioned tools, it will be necessary to construct a modified ACO algorithm to enable its
parallel execution on GPUs with great perfomance and scalability. The initial algorithm model
considered is to calculate ant movements independently. This allows for the simultaneous treatment
of a large set of ants at once. When these ants finish searching for the best path, they return,
and all the data is processed. Then, more ants are sent to calculate the new best path, and this
process continues until the entire database is processed.

It will be necessary to measure the time it takes for to process all paths for each database.
With this data, various charts was produced containing the number of instances being processed
over time. By analyzing these charts, it is possible to identify areas where GPU
processing is more optimized than CPU processing for massive parallel calculations.

\begin{figure}[h!]
\begin{lstlisting}[language=c++]
rocrand_state_xorwow state;
rocrand_init(random_numbers_seed, ant, 0, &state);

int ajk;
for (int j = 0; j < num_inst; j++) {
    ajk = rocrand(&state) % 100;

    if (the_colony[ant*num_inst+j] == -1) {
        if (ajk >= 40)
            the_colony[ant*num_inst+j] = 1;
            else
            the_colony[ant*num_inst+j] = 0;
    }
}
\end{lstlisting}
\caption{randomizing each ant path.}
\label{fig:code1}
\end{figure}

The code snippets present in Figures: \ref{fig:code2}, \ref{fig:code1}, \ref{fig:code3} and \ref{fig:code4} are parts of the most important function $ant\_action$ \footnote{Code can be seen here: \cite{santiagoParallelAco}}.
In Figure \ref{fig:code1} a code snippets was built to make all the ants choose a path to go through all the data available.
This algorithm by definition uses randomized chance for choosing the next path.
rocRAND \cite{rocrand} was used to produce pseudo random numbers for all the ants independently.
Considering that all GPU threads need to get different values for all the different ants. So each ant
can take a different path to the goal. The last part is a measurements of algorithms accuracy.

\begin{figure}[h!]
\begin{lstlisting}[language=c++]
for (int inst_tst = 0; inst_tst < num_inst_test; inst_tst++);
\end{lstlisting}
\caption{loop for each ant to select the next path to take.}
\label{fig:code2}
\end{figure}

The two next code snippets: Figure \ref{fig:code3} and Figure \ref{fig:code4} are inside the \emph{for} in Figure \ref{fig:code2}.
This \emph{for} runs through all the instances available in the dataset and choose
a random path to take. The ant with the best accuracy found will return its value
to be displayed.

\begin{figure}[h!]
\begin{lstlisting}[language=c++]
// KNN process
float menor_distance = FLT_MAX;
float current_class = -1.0f;

for (int inst_select = 0; inst_select < num_inst; inst_select++) {
    if ((ant*num_inst+inst_select) < num_inst && the_colony[ant*num_inst+inst_select] == 1) {
        float dist = distance(&tst_matrix[inst_tst*num_attr_tst], &matrix[inst_select*num_attr], num_attr_tst-1);
        if (dist < menor_distance) {
            menor_distance = dist;
            current_class = matrix[inst_select*num_attr+num_attr-1];
        }
    }
}
if (current_class == tst_matrix[inst_tst*num_attr_tst+num_attr_tst-1]) {
    matches[num_inst_test*ant+inst_tst] = 1;
} else {
    matches[num_inst_test*ant+inst_tst] = 0;
}
\end{lstlisting}
\caption{testing distances to check accuracy.}
\label{fig:code3}
\end{figure}


The code shown in Figure \ref{fig:code4} is used to check what would be if an ant took all the paths possible. That's important to check
if the algorithm accuracy is valid. Tooking all the paths makes the accuracy smaller than the best ant.

\begin{figure}[h!]
\begin{lstlisting}[language=c++]
// KNN of a "perfect" ant
float menor_distance_conjunto_completo = FLT_MAX;
float current_class_conjunto_completo = -1.0f;

for (int inst_select = 0; inst_select < num_inst; inst_select++) {
    float dist_completo = distance(&tst_matrix[inst_tst*num_attr_tst], &matrix[inst_select*num_attr], num_attr_tst-1);

    if (dist_completo < menor_distance_conjunto_completo) {
        menor_distance_conjunto_completo = dist_completo;
        current_class_conjunto_completo = matrix[inst_select*num_attr+num_attr-1];
    }
}
if (current_class_conjunto_completo == tst_matrix[inst_tst*num_attr_tst+num_attr_tst-1]) {
    matches_completo[num_inst_test*ant+inst_tst] = 1;
} else {
    matches_completo[num_inst_test*ant+inst_tst] = 0;
}
\end{lstlisting}
\caption{ant choosing all possible paths.}
\label{fig:code4}
\end{figure}


To compile the code the flags \emph{-O2 -lrocrand -Wall -pedantic} were used \ref{fig:code5}.
The CPU version used to compare the code is available inside the examples folder \footnote{This folder is available through this link \href{https://github.com/LucasSnatiago/Ant-Colony-ROCm/tree/main/examples}{Examples of Ant Colony ROCm - Github}.} and compiled
using \emph{-O2 -fopt-info-vec-optimized -fopenmp -lm} \ref{fig:code5}. So the optimizion level used
was the same.

\begin{figure}[h!]
\begin{lstlisting}
CPU: gcc -O2 -fopt-info-vec-optimized -fopenmp acopar3otim.c -o acopar3otimCPU -lm
GPU: hipcc -O2 -lrocrand -Wall -pedantic acopar3otim.cpp -o acopar3otim
\end{lstlisting}
\caption{compilation flags used.}
\label{fig:code5}
\end{figure}

It's important to notice that the CPU version actually runs in parallel all
regions that has no data dependency. It uses device map as well, so if the computer
supports running a loop inside any coprocessor available it is going to split the work between CPU and this
secundary device.
It uses openmp as the multithreading library. Openmp has the advantage of being easy
to program using it, as simples as adding the correct directives above the code, but
has the disavantage of not being able to control the data flow inside the coprocessor as much
as coding directly in ROCm. Another big problem is, the code is not written thinking
of the parallelism, so there are too many memory synchronization between threads
that makes the code much slower.

The original algorithm, developed by a fellow student at PUC Minas, is accessible via the following link: \footnote{See \href{https://github.com/LucasSnatiago/Ant-Colony-ROCm/blob/main/examples/acopar3otim.c}{acopar3otim.c - Github} for the algorithm details.}. This particular version of the algorithm is parallelized and operates without the use of pheromones. A scholarly article is currently in preparation to highlight the advantages of this version compared to its sequential counterpart that utilizes pheromones. To ensure consistency and integrity of the original design, the algorithm itself was not altered. Instead, only its computational process was adapted for execution on the GPU, a modification implemented as follows:
\begin{enumerate}
    \item Collect all the necessary data.
    \item Move all the data from the host to the device.
    \item Free all the host memory \footnote{The data on the host will be desynchronized from the device, no need to keep a copy on the host.}.
    \item Process all the data independently:
    \subitem Each device thread can use its own auxiliar vector.
    \subitem The colony vector will be the same for every thread, but no thread can access the same vector adresses.
    \item After one thread finishing its computation, wait for all others to finish.
    \item Move all data from device back to host.
    \item Calculate how accurate was the best choosen ant.
    \item Calculate how accurate was the default ant (the one that took all paths).
    \item Free all host memory.
    \item Free all device memory.
\end{enumerate}
