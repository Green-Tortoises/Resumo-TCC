\section{Introduction}

Algorithms for selecting the best instances are well known in the field of computing, such as \emph{k-means} or \emph{Support Vector Machine}. They are of great importance when it is necessary to choose similar elements within a larger group of elements. In this paper, we will address the parallelization of the algorithm \emph{``Instance Selection with Ant Colony Optimization''} (ACO), a more efficient algorithm than both for instance selection.

This algorithm is based on the idea of an ant colony that separates in search of the best possible path to a certain goal. The path chosen by a first ant becomes more likely to be chosen by subsequent ones.

This algorithm is designed to choose random paths until it finds a destination, in the end the best path out of all displaced by the ants will be returned. There is no guarantee that the optimal result will be found. Each execution of the algorithm generates different returns, due to its randomness. Its computational cost is higher than linear. As a result, for each new ant, the algorithm takes a significantly larger amount of processing than the previous amount.

\section{Problem}

An algorithm with good results and low computational cost is of paramount importance. Greater efficiency and lower resource use makes an algorithm a good choice for a larger number of people. So, would it be possible to make an instance selection algorithm parallel to make better use of the available computational resources?

\subsection{Objectives}

The initial idea proposed is to make this AI more efficient. With the correct directives, it is possible that there will be a considerable gain in performance. The algorithm has many regions with large matrix calculations that can be accelerated, as well as operations that can be optimized for GPU-efficient computing platforms, such as ROCm.

\subsection{Main Objective}

Therefore, the main objective of this monograph is to develop and evaluate the parallel form of the algorithm "Instance Selection with Ant Colony Optimization" using ROCm. This algorithm has large parallelizable regions, due to the excess of calculations using matrices and, because of this, the use of GPUs would be an appropriate choice.

\subsection{Specific Objectives}

Some objectives need to be resolved beforehand in order to achieve the main objective. One first is to modify the original algorithm to support GPU parallelism, such as moving the dataframes to the VRAM as soon as the algorithm starts and keeping the copy of the data synchronized between the host and the device whenever possible. Another point is to check the compatibility of the libraries used for the application to work with the use of CUDA. Finally, add parallelism directives to the code in the functions that support being operated on GPU.

\subsection{Expected Contributions}

It is expected that after the parallel version of the algorithm is made, there will be some performance gain compared to the sequential version. With this, it will be possible to execute the same datasets more quickly and will allow people who have AMD and Nvidia GPUs to benefit from better use of their computational resources. An article is already being produced by PUC Minas that will directly benefit from a parallel version of this algorithm.

\subsection{Organization of this Monograph}

This monograph is organized as follows: section 2 presents a literature review, section 3 presents related work, section 4 reviews the operation of the "ANT-IS" algorithm, section 5 describes the methodology used, and finally, a section of conclusions.
