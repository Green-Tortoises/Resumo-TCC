Eu, Lucas Santiago de Oliveira, juntamente ao meu orientador, Henrique Cota de Freitas, estamos
desenvolvendo uma forma paralelizada de um algoritmo de IA chamado \emph{``Instance Selection with Ant Colony Optimization''}. Esse algoritmo tem como
base a ideia de uma colônia de formigas que se separam em busca de encontrar o melhor caminho
possível até um certo objetivo. O caminho escolhido por uma primeira formiga torna-se mais provável de ser escolhido
pelas subsequentes.

Esse algoritmo é projetado para escolher caminhos aleatórios até encontrar um destino, no final o melhor caminho dentro de todos deslocados pelas formigas
será o retornado. Não há garantias de que o resultado ótimo será encontrado. Cada execução do algoritmo gera retornos diferentes, por conta
de sua aleatoriedade. Seu custo computacional é superior ao linear. Com isso, para cada nova formiga o algoritmo toma uma quantidade
significativamente maior de processamento que a quantidade anterior.

A ideia inicial proposta é tornar essa IA mais eficiente. Com as diretivas corretas, é possível que haja um ganho considerável de 
desempenho. O algoritmo possuí muitas regiões com grandes cálculos matriciais que podem ser acelerados, tanto quanto operações que
podem ser otimizadas para plataformas de computação eficiêntes em GPU, como por exemplo, CUDA.
