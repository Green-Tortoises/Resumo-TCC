\section{Introduction}

Instance selection algorithms are pivotal in machine learning applications, including techniques like \emph{K-Means} clustering and \emph{Support Vector Machines} (SVMs). These algorithms are essential for identifying and grouping similar elements within extensive datasets. This paper presents a parallel algorithm modeled after the \emph{``Instance Selection with Ant Colony Optimization''} (ACO) approach, showcasing enhanced efficiency in selecting instances.

ACO is an optimization algorithm inspired by the foraging behavior of ant colonies and is a part of swarm intelligence, a subset of artificial intelligence. In nature, ants find the shortest paths from their colony to food sources by laying down pheromones and following trails with higher pheromone concentrations. This bio-inspired algorithm simulates such behavior.

The ACO algorithm involves a number of artificial ants searching for quality solutions to a problem, each moving through the parameter space representing possible solutions. These ants communicate indirectly by modifying the environment (laying down a pheromone trail), which influences the probability of other ants following the same path. Over time, beneficial paths accumulate more pheromones and therefore have a higher probability of being chosen, allowing the collective of ants to find optimal or near-optimal solutions.

ACO has been effectively applied to a variety of combinatorial optimization problems, such as the traveling salesman problem, routing in telecommunication networks, and scheduling. Its main strengths are the ability to adapt to changes in real-time and to find optimal paths in complex search spaces.

\begin{figure}
    \centering
    \includegraphics[width=0.65\textwidth]{simple_ant.png}
    \caption{A basic visual illustration of the Ant Colony Optimization (ACO) algorithm.}
    \label{fig:simpleant}
\end{figure}

Looking at Figure \ref{fig:simpleant}, it is possible to see how ants behave. First, ants walk through
the data randomly, depositing pheromones. Subsequently, the routes where ants efficiently locate the object become more pheromone-rich compared to lesser-traveled paths. Ultimately, this leads to a significant portion of the colony converging on the most pheromone-dense route.

An algorithm with good results and low computational cost is of paramount importance. Greater efficiency and lower resource use make an algorithm a good choice for a larger number of use cases. So, would it be possible to make an instance selection algorithm parallel to make better use of the available computational resources?

The initial idea proposed is to make this AI more efficient. With the correct directives, it is possible that a considerable gain in performance will be seen. The algorithm has many regions with large matrix calculations that can be accelerated, as well as operations that can be optimized for GPU-efficient computing platforms, such as ROCm \cite{rocm}.

Therefore, the primary aim of this paper is to develop and assess a parallel version of the \emph{``Instance Selection with Ant Colony Optimization''} algorithm, utilizing ROCm. This adaptation will involve the removal of pheromone trails to eliminate data dependencies among the ants.
This algorithm has large parallelizable regions, due to the excess of calculations using matrices and, because of this, the use of GPUs is an appropriate choice.

Significant enhancements are evident in this version of the code. These performance gains are a direct result of more efficient utilization of the available computational resources. The results start from dividing the time in half up to a tenth of the original’s code time. An article currently in development at PUC Minas will greatly benefit from the parallel version of this algorithm.

This paper is organized as follows: Section 2 presents a literature review and related works, Section 3 show all the materials used, Section 4 reviews the operation of the ``ANT-IS" algorithm, Section 5 describes the methodology used, and finally, a Section of conclusions.
