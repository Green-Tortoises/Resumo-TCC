\section{Introduction}

Algorithms for selecting optimal instances are well-established in the field of computing, such as \emph{k-means} and \emph{Support Vector Machine} (SVM). They play a crucial role in identifying similar elements within a larger set. This paper discusses a parallel algorithm inspired by the \emph{``Instance Selection with Ant Colony Optimization''} (ACO) algorithm, demonstrating greater efficiency in instance selection compared to both k-means and SVM.

The ACO algorithm is based on the idea of an ant colony that separates in search of the best possible path to a certain goal, leaving pheromones on the path so the next ant can be more likely to choose this one. This algorithm is designed to choose random paths until it finds a destination, in the end the best path out of all displaced by the ants will be returned. There is no guarantee that the optimal result will be found. Each execution of the algorithm generates different returns, due to its randomness. Its computational cost is higher than linear. As a result, for each new ant, the algorithm takes a significantly larger amount of processing than the previous one.

An algorithm with good results and low computational cost is of paramount importance. Greater efficiency and lower resource use make an algorithm a good choice for a larger number of use cases. So, would it be possible to make an instance selection algorithm parallel to make better use of the available computational resources?

The initial idea proposed is to make this AI more efficient. With the correct directives, it is possible that a considerable gain in performance will be seen. The algorithm has many regions with large matrix calculations that can be accelerated, as well as operations that can be optimized for GPU-efficient computing platforms, such as ROCm \cite{rocm}.

Therefore, the primary aim of this monograph is to develop and assess a parallel version of the \emph{``Instance Selection with Ant Colony Optimization''} algorithm, utilizing ROCm. This adaptation will involve the removal of pheromone trails to eliminate data dependencies among the ants.
This algorithm has large parallelizable regions, due to the excess of calculations using matrices and, because of this, the use of GPUs is an appropriate choice.

Some objectives need to be resolved beforehand in order to achieve the main objective. One first is to modify the original algorithm to support GPU parallelism, such as moving the dataframes to the VRAM as soon as the algorithm starts and keeping the copy of the data synchronized between the host and the device only after all the calculations are done.
As said before, a important change to the original algorithm was made. The ant pheromones was removed making all ants completely independent.
Another point is to check the compatibility of the libraries used for the application to work with the use of ROCm. Finally, modify sections of the code within functions that are suitable for GPU execution.

Significant enhancements are evident in this version of the code. These performance gains are a direct result of more efficient utilization of the available computational resources. An article currently in development at PUC Minas will greatly benefit from the parallel version of this algorithm.

This monograph is organized as follows: Section 2 presents a literature review and related works, Section 3 show all the materials used, Section 4 reviews the operation of the ``ANT-IS" algorithm, Section 5 describes the methodology used, and finally, a Section of conclusions.
